%============ M A S T E R    T H E S I S (main document) ================

\documentclass[a4paper, 12pt]{article}

%==================== P a c k a g e s ================

\usepackage[utf8]{inputenc}
\usepackage[T1]{fontenc}
\usepackage[english]{babel}
\usepackage{amsmath}
\usepackage{amssymb}
\usepackage{amsfonts}
\usepackage{mathtools}
\usepackage{graphicx}
\usepackage{caption}
\usepackage{subcaption}
\usepackage{float}
\usepackage{wrapfig}
\usepackage{array}
\usepackage{booktabs}
\usepackage{multirow}
\usepackage{colortbl}
\usepackage{xcolor}
\usepackage{geometry}
\usepackage{fancyhdr}
\usepackage{hyperref}
\usepackage{bookmark}
\usepackage{setspace}
\usepackage{listings}
\usepackage{algorithm}
\usepackage{algpseudocode}
\usepackage{siunitx}
\usepackage{csquotes}
\usepackage{enumitem}
\usepackage{lipsum}
\usepackage[backend=biber,style=apa,sorting=nyt]{biblatex}

%==================== I m a g e s ====================
\graphicspath{{Images/}}

%==================== C o l o r s ====================
\definecolor{darkgreen}{rgb}{0.0, 0.5, 0.0}

%==================== B i b l i o g r a p h y ====================
\addbibresource{master.bib}

%==================== S e t t i n g s ====================
\geometry{left=1in, right=1in, top=1in, bottom=1in}
\setlength{\headheight}{14.49998pt}
\addtolength{\topmargin}{-2.49998pt}

\onehalfspacing

\hypersetup{
    colorlinks=true,
    linkcolor=black,
    citecolor=black,
    urlcolor=black
}

\newcommand{\E}{\mathbb{E}}
\newcommand{\Var}{\mathrm{Var}}

\setlength{\parindent}{1.25cm}
\setlength{\parskip}{0.25cm}

%=========================== D O C U M E N T ========================
\begin{document}

\pagestyle{fancy}
\fancyhf{}
\lhead{\textsc{Lucas Dubois}}
\rhead{\textsc{\today}}
\cfoot{\thepage}
\renewcommand{\headrulewidth}{0pt}


%------------ Title Page --------------
\begin{titlepage}
\vspace{12cm}
\par
\textit{Final Project for the course of Macroeconomics and Finance}
\par
\textit{HEC Liège}
\par
\textit{Prof. Hambuckers \& Prof. De Backer}
\par
\textit{2025-2026}
\begin{center}
\rule{\linewidth}{1pt}
\vspace{0,5cm}
\vspace{0.75cm}
\begin{spacing}{1.75}
{\LARGE \textbf{\textsc{Report:} \textit{A term-structure model for bonds and a SVAR analysis of monetary policy shocks in Germany}}}
\vspace{0.75cm}
\end{spacing}
\rule{\linewidth}{1pt}
\vspace{0,5cm}
\\
\textsc{{\large MYRIAM LAMBORELLE \& LUCAS DUBOIS }}
\end{center}
\vspace{0.5cm}
\begin{abstract}
    \hspace{1cm} As seen in the course, the New Keynesian (NK) model serves as a fundamental framework in macroeconomic theory and the formulation of monetary policy. 
At its core lies the concept of \textit{divine coincidence}, which suggests that stabilizing inflation successfully stabilizes the output gap that influences welfare. 
This finding significantly eases the challenges faced by central banks since it clearly implies that prioritizing inflation control is enough to reach market efficiency.
However, in their 2007 paper, Blanchard and Galí question this assumption by integrating the notion of \textit{real wage rigidities } in the NK framework. 
Their findings reveal that once we take these frictions into account, the divine coincidence disappears.
\end{abstract}

\vspace{3cm}

\begin{center}
{\Large \today}
\end{center}
\end{titlepage}
\tableofcontents
\newpage

%-------------- Main Content --------------
Hello
\end{document}